\titlespacing*{\part}{0pt}{-20pt}{30pt} % to fix table and plot together on the first page (SEE RESTORE AT BOTTOM)
\titlespacing*{\chapter}{0pt}{-10pt}{30pt}

\part{Generative Learning Algorithms}
\label{part:generative_algs}

\marginnote{From CS229 Spring 2021, Andrew Ng, Moses Charikar, \& Christopher R\'e, Stanford University.}

So far, we've mainly been talking about learning algorithms that model
$p(y \mid x;\theta)$, the conditional distribution of $y$ given $x$. For instance, logistic
regression modeled $p(y \mid x;\theta)$ as $h_\theta (x) = g(\theta^\top x)$ where $g$ is the sigmoid function.
In these notes, we'll talk about a different type of learning algorithm.

Consider a classification problem in which we want to learn to distinguish
between elephants ($y = 1$) and dogs ($y = 0$), based on some features of
an animal. Given a training set, an algorithm like logistic regression or
the perceptron algorithm (basically) tries to find a straight line---that is, a
decision boundary---that separates the elephants and dogs. Then, to classify
a new animal as either an elephant or a dog, it checks on which side of the
decision boundary it falls, and makes its prediction accordingly.

Here's a different approach. First, looking at elephants, we can build a
model of what elephants look like. Then, looking at dogs, we can build a
separate model of what dogs look like. Finally, to classify a new animal, we
can match the new animal against the elephant model, and match it against
the dog model, to see whether the new animal looks more like the elephants
or more like the dogs we had seen in the training set.

Algorithms that try to learn $p(y \mid x)$ directly (such as logistic regression),
or algorithms that try to learn mappings directly from the space of inputs $\mathcal X$
to the labels $\{0,1\}$, (such as the perceptron algorithm) are called \textbf{discriminative}
learning algorithms. Here, we'll talk about algorithms that instead
try to model $p(x \mid y)$ (and $p(y)$). These algorithms are called \textbf{generative}
learning algorithms. For instance, if $y$ indicates whether an example is a
dog ($0$) or an elephant ($1$), then $p(x \mid y = 0)$ models the distribution of dogs'
features, and $p(x \mid y = 1)$ models the distribution of elephants' features.

After modeling $p(y)$ (called the \textbf{class priors}) and $p(x \mid y)$, our algorithm
can then use Bayes rule to derive the posterior distribution on $y$ given $x$:
\begin{equation}
    p(y \mid x) = \frac{p(x \mid y)p(y)}{p(x)}    
\end{equation}
Here, the denominator is given by $p(x) = p(x \mid y = 1)p(y = 1) + p(x \mid y =
0)p(y = 0)$ (you should be able to verify that this is true from the standard
properties of probabilities), and thus can also be expressed in terms of the
quantities $p(x \mid y)$ and $p(y)$ that we've learned. Actually, if we're calculating % TYPO: were
$p(y \mid x)$ in order to make a prediction, then we don't actually need to calculate
the denominator, since
\begin{align*}
    \argmax_y p(y \mid x) &= \argmax_y \frac{p(x \mid y)p(y)}{p(x)}\\
                          &= \argmax_y p(x \mid y)p(y).
\end{align*}

{\let\cleardoublepage\relax \chapter{Gaussian discriminant analysis}}
The first generative learning algorithm that we'll look at is Gaussian discriminant
analysis (GDA). In this model, we'll assume that $p(x \mid y)$ is distributed
according to a multivariate normal distribution. Let's talk briefly about the
properties of multivariate normal distributions before moving on to the GDA
model itself.

The multivariate normal distribution in $d$-dimensions, also called the multivariate
Gaussian distribution, is parameterized by a \textbf{mean vector} $\mu \in \mathbb{R}^d$
and a covariance matrix $\Sigma \in \mathbb{R}^{d \times d}$, where $\Sigma \ge 0$ is symmetric and positive
semi-definite. Also written ``$\mathcal N(\mu,\Sigma)$'', its density is given by:
\begin{equation}
p(x; \mu,\Sigma) = \frac{1}{(2\pi)^{d/2} |\Sigma|^{1/2}} \exp\left( -\frac{1}{2}(x - \mu)^\top \Sigma^{-1}(x - \mu) \right)
\end{equation}
In the equation above, ``$|\Sigma|$'' denotes the determinant of the matrix $\Sigma$.

For a random variable $X$ distributed $\mathcal N(\mu,\Sigma)$, the mean is (unsurprisingly)
given by $\mu$:
\begin{equation}
    \mathbb{E}[X] = \int_x x p(x; \mu, \Sigma) dx = \mu
\end{equation}

The \textbf{covariance} of a vector-valued random variable $Z$ is defined as $\operatorname{Cov}(Z) =
\mathbb E[(Z - E[Z])(Z - E[Z])^\top]$. This generalizes the notion of the variance of a
real-valued random variable. The covariance can also be defined as $\operatorname{Cov}(Z) =
\mathbb E[Z Z^\top] - (\mathbb E[Z])(\mathbb E[Z])^\top$. (You should be able to prove to yourself that these
two definitions are equivalent.) If $X \sim \mathcal N(\mu,\Sigma)$, then
\begin{equation}
    \operatorname{Cov}(X) = \Sigma.
\end{equation}

Here are some examples of what the density of a Gaussian distribution
looks like:
% TODO: multivariate surface plots

The left-most figure shows a Gaussian with mean zero (that is, the $2\times1$
zero-vector) and covariance matrix $\Sigma = I$ (the $2\times2$ identity matrix). A
Gaussian with zero mean and identity covariance is also called the \textbf{standard normal
distribution}. The middle figure shows the density of a Gaussian with
zero mean and $\Sigma = 0.6I$; and in the rightmost figure shows one with, $\Sigma = 2I$.
We see that as $\Sigma$ becomes larger, the Gaussian becomes more ``spread-out,''
and as it becomes smaller, the distribution becomes more ``compressed.''

Let's look at some more examples.
% TODO: multivariate surface plots

The figures above show Gaussians with mean $0$, and with covariance
matrices respectively:
\begin{equation}
    \Sigma = \begin{bmatrix}
    1 & 0\\
    0 & 1
    \end{bmatrix};\quad
    \Sigma = \begin{bmatrix}
    1 & 0.5\\
    0.5 & 1
    \end{bmatrix};\quad
    \Sigma = \begin{bmatrix}
    1 & 0.8\\
    0.8 & 1
    \end{bmatrix}.
\end{equation}
The leftmost figure shows the familiar standard normal distribution, and we
see that as we increase the off-diagonal entry in $\Sigma$, the density becomes more
``compressed'' towards the $45^\circ$ line (given by $x_1 = x_2$). We can see this more
clearly when we look at the contours of the same three densities:
\begin{figure}
    \begin{jlcode}
    p = let
        Σs = [([0,0],[1 0; 0 1]),
              ([0,0],[1 0.5; 0.5 1]),
              ([0,0],[1 0.8; 0.8 1])]
        mv_gaussian_plot(Σs; heatmap=false)
    end
    plot(p)
    \end{jlcode}
    \begin{center}
        \plot{fig/gaussian1}
    \end{center}
\end{figure}


Here's one last set of examples generated by varying $\Sigma$:
\begin{figure}
    \begin{jlcode}
    p = let
        Σs = [([0,0],[3 0.8; 0.8 3]),
              ([0,0],[1 -0.5; -0.5 1]),
              ([0,0],[1 -0.8; -0.8 1])]
        mv_gaussian_plot(Σs; heatmap=false)
    end
    plot(p)
    \end{jlcode}
    \begin{center}
        \plot{fig/gaussian2}
    \end{center}
\end{figure}

The plots above used, respectively, % DIFF: reordered so the Sigmas "match" the previous values.
\begin{equation}
    \Sigma = \begin{bmatrix}
    3 & 0.8\\
    0.8 & 1
    \end{bmatrix};\quad
    \Sigma = \begin{bmatrix}
    1 & -0.5\\
    -0.5 & 1
    \end{bmatrix};\quad
    \Sigma = \begin{bmatrix}
    1 & -0.8\\
    -0.8 & 1
    \end{bmatrix}.
\end{equation}
From the leftmost and middle figures, we see that by decreasing the off-
diagonal elements of the covariance matrix, the density now becomes ``compressed''
again, but in the opposite direction. Lastly, as we vary the parameters,
more generally the contours will form ellipses (the rightmost figure
showing an example).

As our last set of examples, fixing $\Sigma = I$, by varying $\mu$, we can also move
the mean of the density around.

The figures above were generated using $\Sigma = I$, and respectively
\begin{equation}
    \mu = \begin{bmatrix}
    1\\
    0
    \end{bmatrix};\quad
    \mu = \begin{bmatrix}
    -0.5\\
    0
    \end{bmatrix};\quad
    \mu = \begin{bmatrix}
    -1\\
    -1.5
    \end{bmatrix}.    
\end{equation}



\section{The Gaussian Discriminant Analysis model}
When we have a classification problem in which the input features $x$ are
continuous-valued random variables, we can then use the Gaussian Discriminant
Analysis (GDA) model, which models $p(x \mid y)$ using a multivariate normal
distribution. The model is:
\begin{align}
    y &\sim \operatorname{Bernoulli}(\phi)\\
    x \mid y = 0 &\sim \mathcal N(\mu_0 ,\Sigma)\\
    x \mid y = 1 &\sim \mathcal N(\mu_1 ,\Sigma)
\end{align}
Writing out the distributions, this is:
\begin{align}
    p(y) &= \phi^y (1 - \phi)^{1-y}\\
    p(x \mid y = 0) &= \frac{1}{(2\pi)^{d/2} |\Sigma|^{1/2}} \exp\left(-\frac{1}{2} (x - \mu_0)^\top \Sigma^{-1}(x-\mu_0) \right)\\
    p(x \mid y = 1) &= \frac{1}{(2\pi)^{d/2} |\Sigma|^{1/2}} \exp\left(-\frac{1}{2} (x - \mu_1)^\top \Sigma^{-1}(x-\mu_1) \right)    
\end{align}
Here, the parameters of our model are $\phi$, $\Sigma$, $\mu_0$ and $\mu_1$. (Note that while
there're two different mean vectors $\mu_0$ and $\mu_1$, this model is usually applied
using only one covariance matrix $\Sigma$.) The log-likelihood of the data is given
by
\begin{align}
    \ell(\phi,\mu_0 ,\mu_1 ,\Sigma) &= \log \prod_{i=1}^n p(x^{(i)}, y^{(i)}; \phi, \mu_o, \mu_1, \Sigma)\\
                                    &= \log \prod_{i=1}^n p(x^{(i)} \mid y^{(i)}; \phi, \mu_o, \mu_1, \Sigma)p(y^{(i)}; \phi).    
\end{align}
By maximizing $\ell$ with respect to the parameters, we find the maximum likelihood
estimate of the parameters (see problem set 1) to be:
\begin{align}
    \phi &= \frac{1}{n} \sum_{i=1}^n \mathbb{1}\{y^{(i)}=1\}\\
    \mu_0 &= \frac{\sum_{i=1}^n \mathbb{1}\{y^{(i)}=0\}x^{(i)}}{\sum_{i=1}^n \mathbb{1}\{y^{(i)}=0\}}\\
    \mu_1 &= \frac{\sum_{i=1}^n \mathbb{1}\{y^{(i)}=1\}x^{(i)}}{\sum_{i=1}^n \mathbb{1}\{y^{(i)}=1\}}\\
    \Sigma &= \frac{1}{n} \sum_{i=1}^n (x^{(i)} - \mu_{y^{(i)}}) (x^{(i)} - \mu_{y^{(i)}})^\top
\end{align}

Pictorially, what the algorithm is doing can be seen in as follows:
% TODO: diagram

Shown in the figure are the training set, as well as the contours of the
two Gaussian distributions that have been fit to the data in each of the
two classes. Note that the two Gaussians have contours that are the same
shape and orientation, since they share a covariance matrix $\Sigma$, but they have
different means $\mu_0$ and $\mu_1$. Also shown in the figure is the straight line
giving the decision boundary at which $p(y = 1 \mid x) = 0.5$. On one side of
the boundary, we'll predict $y = 1$ to be the most likely outcome, and on the
other side, we'll predict $y = 0$.

\section{Discussion: GDA and logistic regression}
The GDA model has an interesting relationship to logistic regression. If we
view the quantity $p(y = 1 \mid x;\phi,\mu_0 ,\mu_1 ,\Sigma)$ as a function of $x$, we'll find that it
can be expressed in the form
\begin{equation}
    p(y = 1 \mid x;\phi,\Sigma,\mu_0 ,\mu_1 ) = \frac{1}{1 + \exp(-\theta^\top x)},
\end{equation}
where $\theta$ is some appropriate function of $\phi,\Sigma,\mu_0 ,\mu_1$.\footnote{
This uses the convention of redefining the $x^{(i)}$'s on the right-hand-side to be $(d + 1)$-
dimensional vectors by adding the extra coordinate $x^{(i)}_0 = 1$; see problem set 1.}
This is exactly the form that logistic regression---a discriminative algorithm---used to model
$p(y = 1 \mid x)$.

When would we prefer one model over another? GDA and logistic regression
will, in general, give different decision boundaries when trained on the
same dataset. Which is better?

We just argued that if $p(x \mid y)$ is multivariate Gaussian (with shared $\Sigma$), % TYPO: uppercase Gaussian
then $p(y \mid x)$ necessarily follows a logistic function. The converse, however,
is not true; i.e., $p(y \mid x)$ being a logistic function does not imply $p(x \mid y)$ is
multivariate Gaussian. This shows that GDA makes \textit{stronger} modeling assumptions
about the data than does logistic regression. It turns out that
when these modeling assumptions are correct, then GDA will find better fits
to the data, and is a better model. Specifically, when $p(x \mid y)$ is indeed Gaussian
(with shared $\Sigma$), then GDA is \textbf{asymptotically efficient}. Informally,
this means that in the limit of very large training sets (large $n$), there is no
algorithm that is strictly better than GDA (in terms of, say, how accurately
they estimate $p(y \mid x)$). In particular, it can be shown that in this setting,
GDA will be a better algorithm than logistic regression; and more generally,
even for small training set sizes, we would generally expect GDA to be better. % TYPO: missing "be"

In contrast, by making significantly weaker assumptions, logistic regression
is also more \textit{robust} and less sensitive to incorrect modeling assumptions.
There are many different sets of assumptions that would lead to $p(y \mid x)$ taking
the form of a logistic function. For example, if $x \mid y = 0 \sim \operatorname{Poisson}(\lambda_0)$, and
$x \mid y = 1 \sim \operatorname{Poisson}(\lambda_1)$, then $p(y \mid x)$ will be logistic. Logistic regression will
also work well on Poisson data like this. But if we were to use GDA on such
data---and fit Gaussian distributions to such non-Gaussian data---then the
results will be less predictable, and GDA may (or may not) do well.

To summarize: GDA makes stronger modeling assumptions, and is more
data efficient (i.e., requires less training data to learn ``well'') when the modeling
assumptions are correct or at least approximately correct. Logistic
regression makes weaker assumptions, and is significantly more robust to
deviations from modeling assumptions. Specifically, when the data is indeed
non-Gaussian, then in the limit of large datasets, logistic regression will
almost always do better than GDA. For this reason, in practice logistic regression
is used more often than GDA. (Some related considerations about
discriminative vs. generative models also apply for the Naive Bayes algorithm
that we discuss next, but the Naive Bayes algorithm is still considered
a very good, and is certainly also a very popular, classification algorithm.)

\chapter{Naive Bayes}
In GDA, the feature vectors $x$ were continuous, real-valued vectors. Let's
now talk about a different learning algorithm in which the $x_j$'s are discrete-valued.

For our motivating example, consider building an email spam filter using
machine learning. Here, we wish to classify messages according to whether
they are unsolicited commercial (spam) email, or non-spam email. After
learning to do this, we can then have our mail reader automatically filter
out the spam messages and perhaps place them in a separate mail folder.
Classifying emails is one example of a broader set of problems called \textbf{text
classification}.

Let's say we have a training set (a set of emails labeled as spam or non-
spam). We'll begin our construction of our spam filter by specifying the
features $x_j$ used to represent an email.

We will represent an email via a feature vector whose length is equal to
the number of words in the dictionary. Specifically, if an email contains the
$j$-th word of the dictionary, then we will set $x_j = 1$; otherwise, we let $x_j = 0$.
For instance, the vector
\begin{equation*}
    x = \begin{bmatrix}
    1\\0\\0\\\vdots\\1\\\vdots\\0
    \end{bmatrix}\quad
    \begin{matrix}
    \text{a}\\\text{aardvark}\\\text{aardwolf}\\\vdots\\\text{buy}\\\vdots\\\text{zygmurgy}
    \end{matrix}
\end{equation*}
is used to represent an email that contains the words ``a'' and ``buy,'' but not
``aardvark,'' ``aardwolf'' or ``zygmurgy.''\footnote{
Actually, rather than looking through an English dictionary for the list of all English
words, in practice it is more common to look through our training set and encode in our
feature vector only the words that occur at least once there. Apart from reducing the
number of words modeled and hence reducing our computational and space requirements,
this also has the advantage of allowing us to model/include as a feature many words
that may appear in your email (such as ``cs229'') but that you won't find in a dictionary.
Sometimes (as in the homework), we also exclude the very high frequency words (which
will be words like ``the,'' ``of,'' ``and''; these high frequency, ``content free'' words are called
\textbf{stop words}) since they occur in so many documents and do little to indicate whether an
email is spam or non-spam.} The set of words encoded into the
feature vector is called the \textbf{vocabulary}, so the dimension of $x$ is equal to
the size of the vocabulary.

Having chosen our feature vector, we now want to build a generative
model. So, we have to model $p(x \mid y)$. But if we have, say, a vocabulary of
50000 words, then $x \in \{0,1\}^{50000}$ ($x$ is a 50000-dimensional vector of 0's and
1's), and if we were to model $x$ explicitly with a multinomial distribution over
the $2^{50000}$ possible outcomes, then we'd end up with a $(2^{50000} - 1)$-dimensional
parameter vector. This is clearly too many parameters.

To model $p(x \mid y)$, we will therefore make a very strong assumption. We will
assume that the $x_i$'s are conditionally independent given $y$. This assumption
is called the \textbf{Naive Bayes (NB) assumption}, and the resulting algorithm is
called the \textbf{Naive Bayes classifier}. For instance, if $y = 1$ means spam email;
``buy'' is word 2087 and ``price'' is word 39831; then we are assuming that if
I tell you $y = 1$ (that a particular piece of email is spam), then knowledge
of $x_{2087}$ (knowledge of whether ``buy'' appears in the message) will have no
effect on your beliefs about the value of $x_{39831}$ (whether ``price'' appears).
More formally, this can be written $p(x_{2087} \mid y) = p(x_{2087} \mid y, x_{39831})$. (Note that
this is not the same as saying that $x_{2087}$ and $x_{39831}$ are independent, which
would have been written ``$p(x_{2087}) = p(x_{2087} \mid x_{39831})$''; rather, we are only
assuming that $x_{2087}$ and $x_{39831}$ are conditionally independent given $y$.)

We now have:
\begin{align}
    p(&x_1 ,\ldots,x_{50000} \mid y)\\
    &= p(x_1 \mid y)p(x_2 \mid y,x_1 )p(x_3 \mid y,x_1 ,x_2 )\cdots p(x_{50000} \mid y,x_1 ,\ldots,x_{49999} )\\
    &= p(x_1 \mid y)p(x_2 \mid y)p(x_3 \mid y)\cdots p(x_{50000} \mid y)\\
    &= \prod_{j=1}^d p(x_j \mid y)
\end{align}
The first equality simply follows from the usual properties of probabilities,
and the second equality used the NB assumption. We note that even though
the Naive Bayes assumption is an extremely strong assumptions, the resulting
algorithm works well on many problems.

Our model is parameterized by $\phi_{j \mid y=1} = p(x_j = 1 \mid y = 1)$, $\phi_{j \mid y=0} = p(x_j =
1 \mid y = 0)$, and $\phi_y = p(y = 1)$. As usual, given a training set $\{(x^{(i)} ,y^{(i)});i =
1,\ldots,n\}$, we can write down the joint likelihood of the data:
\begin{equation}
    \mathcal L(\phi_y ,\phi_{j \mid y=0}, \phi_{j \mid y=1}) = \prod_{i=1}^n p(x^{(i)}, y^{(i)})
\end{equation}
Maximizing this with respect to $\phi_y$, $\phi_{j \mid y=0}$ and $\phi_{j \mid y=1}$ gives the maximum
likelihood estimates:
\begin{align}
    \phi_{j \mid y=1} &= \frac{\sum_{i=1}^n \mathbb{1}\{x_j^{(i)} = 1 \wedge y^{(i)} = 1 \}}{\sum_{i=1}^n \mathbb{1}\{ y^{(i)} = 1 \}}\\
    \phi_{j \mid y=0} &= \frac{\sum_{i=1}^n \mathbb{1}\{x_j^{(i)} = 1 \wedge y^{(i)} = 0 \}}{\sum_{i=1}^n \mathbb{1}\{ y^{(i)} = 0 \}}\\
    \phi_y &= \frac{\sum_{i=1}^n \mathbb{1}\{y^{(i)} = 1\}}{n}
\end{align}
In the equations above, the ``$\wedge$'' symbol means ``and.'' The parameters have
a very natural interpretation. For instance, $\phi_{j \mid y=1}$ is just the fraction of the
spam $(y = 1)$ emails in which word $j$ does appear.

Having fit all these parameters, to make a prediction on a new example
with features $x$, we then simply calculate
\begin{align}
    p(y = 1 \mid x) &= \frac{p(x \mid y=1)p(y=1)}{p(x)}\\
    &= \frac{\left( \prod_{j=1}^d p(x_j \mid y=1) \right) p(y=1) }{\left( \prod_{j=1}^d p(x_j \mid y=1) \right) p(y=1) + \left( \prod_{j=1}^d p(x_j \mid y=0) \right) p(y=0)},
\end{align}
and pick whichever class has the higher posterior probability.

Lastly, we note that while we have developed the Naive Bayes algorithm
mainly for the case of problems where the features $x_j$ are binary-valued, the
generalization to where $x_j$ can take values in $\{1,2,\ldots,k_j\}$ is straightforward.
Here, we would simply model $p(x_j \mid y)$ as multinomial rather than as Bernoulli.
Indeed, even if some original input attribute (say, the living area of a house,
as in our earlier example) were continuous valued, it is quite common to
discretize it---that is, turn it into a small set of discrete values---and apply
Naive Bayes. For instance, if we use some feature $x_j$ to represent living area,
we might discretize the continuous values as follows:
\begin{table}[!h]
  \centering
  \caption{
    \label{tab:nb_discretize} Discretized living area.
  }
  \begin{tabular}{c|c|c|c|c|c}
    \toprule
    Living area (ft$^2$) & $< 400$ & $400-800$ & $800-1200$ & $1200-1600$ & $> 1600$\\ % DIFF: sq. feet to ft^2
    \midrule
    $x_i$ & 1 & 2 & 3 & 4 & 5\\
    \bottomrule
  \end{tabular}
\end{table}

Thus, for a house with living area 890 square feet, we would set the value
of the corresponding feature $x_j$ to 3. We can then apply the Naive Bayes
algorithm, and model $p(x_j \mid y)$ with a multinomial distribution, as described
previously. When the original, continuous-valued attributes are not well-
modeled by a multivariate normal distribution, discretizing the features and
using Naive Bayes (instead of GDA) will often result in a better classifier.

\section{Laplace smoothing}
The Naive Bayes algorithm as we have described it will work fairly well
for many problems, but there is a simple change that makes it work much
better, especially for text classification. Let's briefly discuss a problem with
the algorithm in its current form, and then talk about how we can fix it.

Consider spam/email classification, and let's suppose that, we are in the
year of 20xx, after completing CS229 and having done excellent work on the
project, you decide around May 20xx to submit work you did to the NeurIPS
conference for publication.\footnote{NeurIPS is one of the top machine learning conferences. The deadline for submitting
a paper is typically in May-June.} Because you end up discussing the conference
in your emails, you also start getting messages with the word ``neurips''
in it. But this is your first NeurIPS paper, and until this time, you had
not previously seen any emails containing the word ``neurips''; in particular
``neurips'' did not ever appear in your training set of spam/non-spam emails.
Assuming that ``neurips'' was the 35000th word in the dictionary, your Naive
Bayes spam filter therefore had picked its maximum likelihood estimates of
the parameters $\phi_{35000 \mid y}$ to be
\begin{align}
    \phi_{35000 \mid y=1} &= \frac{\sum_{i=1}^n \mathbb{1}\{x_{35000}^{(i)} = 1 \wedge y^{(i)} = 1 \}}{\sum_{i=1}^n \mathbb{1}\{ y^{(i)} = 1 \}} = 0\\
    \phi_{35000 \mid y=0} &= \frac{\sum_{i=1}^n \mathbb{1}\{x_{35000}^{(i)} = 1 \wedge y^{(i)} = 0 \}}{\sum_{i=1}^n \mathbb{1}\{ y^{(i)} = 0 \}} = 0,
\end{align}
i.e., because it has never seen ``neurips'' before in either spam or non-spam
training examples, it thinks the probability of seeing it in either type of email
is zero. Hence, when trying to decide if one of these messages containing
``neurips'' is spam, it calculates the class posterior probabilities, and obtains
\begin{align}
    p(y = 1 \mid x) &= \frac{\prod_{j=1}^d p(x_j \mid y=1) p(y=1)}{\prod_{j=1}^d p(x_j \mid y=1) p(y=1) + \prod_{j=1}^d p(x_j \mid y=0) p(y=0)}\\
    &= \frac{0}{0}
\end{align}
This is because each of the terms ``$\prod_{j=1}^d p(x_j \mid y)$'' includes a term $p(x_{35000} \mid y) =
0$ that is multiplied into it. Hence, our algorithm obtains 0/0, and doesn't
know how to make a prediction.

Stating the problem more broadly, it is statistically a bad idea to estimate
the probability of some event to be zero just because you haven't seen
it before in your finite training set. Take the problem of estimating the mean
of a multinomial random variable $z$ taking values in $\{1,\ldots,k\}$. We can parameterize
our multinomial with $\phi_j = p(z = j)$. Given a set of $n$ independent
observations $\{z^{(1)} ,\ldots,z^{(n)}\}$, the maximum likelihood estimates are given by
\begin{equation}
    \phi_j = \frac{\sum_{i=1}^n \mathbb{1}\{z^{(i)} = j\}}{n}.
\end{equation}
As we saw previously, if we were to use these maximum likelihood estimates,
then some of the $\phi_j$'s might end up as zero, which was a problem. To avoid
this, we can use \textbf{Laplace smoothing}, which replaces the above estimate
with
\begin{equation}
    \phi_j = \frac{1 + \sum_{i=1}^n \mathbb{1}\{z^{(i)} = j\}}{k + n}.
\end{equation}
Here, we've added 1 to the numerator, and $k$ to the denominator. Note that
$\sum_{j=1}^k \phi_j = 1$ still holds (check this yourself!), which is a desirable property
since the $\phi_j$'s are estimates for probabilities that we know must sum to 1.
Also, $\phi_j \ne 0$ for all values of $j$, solving our problem of probabilities being
estimated as zero. Under certain (arguably quite strong) conditions, it can
be shown that the Laplace smoothing actually gives the optimal estimator
of the $\phi_j$'s.

Returning to our Naive Bayes classifier, with Laplace smoothing, we
therefore obtain the following estimates of the parameters:
\begin{align}
    \phi_{j \mid y=1} &= \frac{1 + \sum^n_{i=1} \mathbb{1}\{x^{(i)}_j = 1 \wedge y^{(i)} = 1\}}{2 + \sum^n_{i=1} \mathbb{1}\{y ^{(i)} = 1\}}\\
    \phi_{j \mid y=0} &= \frac{1 + \sum^n_{i=1} \mathbb{1}\{x^{(i)}_j = 1 \wedge y^{(i)} = 0\}}{2 + \sum^n_{i=1} \mathbb{1}\{y ^{(i)} = 0\}}
\end{align}
(In practice, it usually doesn't matter much whether we apply Laplace smoothing
to $\phi_y$ or not, since we will typically have a fair fraction each of spam and
non-spam messages, so $\phi_y$ will be a reasonable estimate of $p(y = 1)$ and will
be quite far from 0 anyway.)

\section{Event models for text classification}
To close off our discussion of generative learning algorithms, let's talk about
one more model that is specifically for text classification. While Naive Bayes
as we've presented it will work well for many classification problems, for text
classification, there is a related model that does even better.

In the specific context of text classification, Naive Bayes as presented uses
the what's called the \textbf{Bernoulli event model} (or sometimes \textbf{multi-variate
Bernoulli event model}). In this model, we assumed that the way an email
is generated is that first it is randomly determined (according to the class
priors $p(y)$) whether a spammer or non-spammer will send you your next
message. Then, the person sending the email runs through the dictionary,
deciding whether to include each word $j$ in that email independently and
according to the probabilities $p(x_j = 1 \mid y) = \phi_{j \mid y}$. Thus, the probability of a
message was given by $p(y) \prod_{j=1}^d p(x_j \mid y)$

Here's a different model, called the \textbf{Multinomial event model}. To
describe this model, we will use a different notation and set of features for
representing emails. We let $x_j$ denote the identity of the $j$-th word in the
email. Thus, $x_j$ is now an integer taking values in $\{1,\ldots,|V|\}$, where $|V|$
is the size of our vocabulary (dictionary). An email of $d$ words is now represented
by a vector $(x_1 ,x_2 ,\ldots,x_d )$ of length $d$; note that $d$ can vary for
different documents. For instance, if an email starts with ``A NeurIPS \ldots,''
then $x_1 = 1$ (``a'' is the first word in the dictionary), and $x_2 = 35000$ (if
``neurips'' is the 35000th word in the dictionary).

In the multinomial event model, we assume that the way an email is
generated is via a random process in which spam/non-spam is first determined
(according to $p(y)$) as before. Then, the sender of the email writes the
email by first generating $x_1$ from some multinomial distribution over words
($p(x_1 \mid y)$). Next, the second word $x_2$ is chosen independently of $x_1$ but from
the same multinomial distribution, and similarly for $x_3$, $x_4$, and so on, until
all $d$ words of the email have been generated. Thus, the overall probability of
a message is given by $p(y)
\prod^d_{j=1} p(x_j \mid y)$. Note that this formula looks like the
one we had earlier for the probability of a message under the Bernoulli event
model, but that the terms in the formula now mean very different things. In
particular $x_j \mid y$ is now a multinomial, rather than a Bernoulli distribution.

The parameters for our new model are $\phi_y = p(y)$ as before, $\phi_{k \mid y=1} =
p(x_j = k \mid y = 1)$ (for any $j$) and $\phi_{k \mid y=0} = p(x_j = k \mid y = 0)$. Note that we have
assumed that $p(x_j \mid y)$ is the same for all values of $j$ (i.e., that the distribution
according to which a word is generated does not depend on its position $j$
within the email).

If we are given a training set $\{(x^{(i)} ,y^{(i)} );i = 1,\ldots,n\}$ where $x^{(i)} = (x^{(i)}_1,x^{(i)}_2,\ldots,x^{(i)}_{d_i})$ (here, $d_i$ is the number of words in the $i$-training example), the likelihood of the data is given by
\begin{align}
    \mathcal L(\phi_y ,\phi_{k \mid y=0} ,\phi_{k \mid y=1} ) &= \prod_{i=1}^{n} p(x^{(i)} ,y^{(i)} )\\
    &= \prod_{i=1}^{n}\left( \prod_{j=1}^{d_i} p(x^{(i)}_j \mid y;\phi_{k \mid y=0} ,\phi_{k \mid y=1}) \right) p(y^{(i)} ;\phi_y).
\end{align}
Maximizing this yields the maximum likelihood estimates of the parameters:
\begin{align}
    \phi_{k \mid y=1} &= \frac{\sum^{n}_{i=1}\sum^{d_i}_{j=1} \mathbb{1}\{x^{(i)}_j = k \wedge y^{(i)} = 1\}}{\sum^{n}_{i=1} \mathbb{1}\{y^{(i)} = 1\}d_i}\\
    \phi_{k \mid y=0} &= \frac{\sum^{n}_{i=1}\sum^{d_i}_{j=1} \mathbb{1}\{x^{(i)}_j = k \wedge y^{(i)} = 0\}}{\sum^{n}_{i=1} \mathbb{1}\{y^{(i)} = 0\}d_i}\\
    \phi_y &= \frac{\sum^{n}_{i=1} \mathbb{1}\{y^{(i)}= 1\}}{n}.
\end{align}
If we were to apply Laplace smoothing (which is needed in practice for good
performance) when estimating $\phi_{k \mid y=0}$ and $\phi_{k \mid y=1}$, we add 1 to the numerators
and $|V|$ to the denominators, and obtain:
\begin{align}
    \phi_{k \mid y=1} = \frac{1 + \sum^{n}_{i=1}\sum^{d_i}_{j=1} \mathbb{1}\{x^{(i)}_j = k \wedge y^{(i)} = 1\}}{|V| + \sum^{n}_{i=1} \mathbb{1}\{y^{(i)} = 1\}d_i}\\
    \phi_{k \mid y=0} = \frac{1 + \sum^{n}_{i=1}\sum^{d_i}_{j=1} \mathbb{1}\{x^{(i)}_j = k \wedge y^{(i)} = 0\}}{|V| + \sum^{n}_{i=1} \mathbb{1}\{y^{(i)} = 0\}d_i}
\end{align}
While not necessarily the very best classification algorithm, the Naive Bayes
classifier often works surprisingly well. It is often also a very good ``first thing
to try,'' given its simplicity and ease of implementation.