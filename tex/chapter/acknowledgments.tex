\chapter*{Acknowledgments}
\addcontentsline{toc}{chapter}{Acknowledgments}

This work is taken from the lecture notes for the course \textit{Machine Learning} at Stanford University, CS 229 (\rurl{cs229.stanford.edu}). % TODO: remove extra spaces between " . "
The contributors to the content of this work are Andrew Ng and Christopher R\'e---this collection is simply a typesetting of existing lecture notes with minor modifications and additions of working Julia implementations. We would like to thank the original authors for their contribution. % TODO: Include other contributors from \marginnotes
% The authors wish to thank the many individuals who have provided valuable feedback on early drafts of our manuscript, including
% Adam Adamson,
% Barry Barryson,
% Cindy Cindyson, and
% Daryl Darylson.
% In addition, it has been a pleasure working with our editor from the MIT Press in preparing this manuscript for publication.
In addition, we wish to thank Mykel Kochenderfer and Tim Wheeler for their contribution to the Tufte-Algorithms \LaTeX{} template, based off of \textit{Algorithms for Optimization}.\cite{Kochenderfer2019}
% TODO: AA222/CS361 mention/link?

% The style of this book was inspired by Edward Tufte.
% Among other stylistic elements, we adopted his wide margins and use of small multiples.
% In fact, the typesetting of this book is heavily based on the Tufte-LaTeX package by Kevin Godby, Bil Kleb, and Bill Wood.

% We have also benefited from the various open source packages on which this textbook depends.
% The typesetting of the code is done with the help of pythontex, which is maintained by Geoffrey Poore.
% Plotting is handled by pgfplots, which is maintained by Christian Feuers\"{a}nger.
% The book's color scheme was adapted from the Monokai theme by Jon Skinner of Sublime Text (\url{sublimetext.com}).
% For plots, we use the viridis colormap defined by St\'efan van der Walt and Nathaniel Smith.


\vspace{5ex}
\noindent\textsc{Robert J. Moss}\\
Stanford, Calif.\\
\psetdate\today
\vfill
\noindent Ancillary material is available on the template's webpage:\\
\noindent\url{https://github.com/sisl/textbook_template}

% Based on handouts by Mehran Sahami, Chris Piech, and Lisa Yan.
% Stanford University, CS109, Spring 2020