\titlespacing*{\part}{0pt}{-20pt}{30pt} % to fix table and plot together on the first page (SEE RESTORE AT BOTTOM)
\titlespacing*{\chapter}{0pt}{-10pt}{30pt}

\part{Factor Analysis}
\label{part:factor_analysis}

\marginnote{From CS229 Spring 2021, Andrew Ng, Moses Charikar \& Christopher R\'e, Stanford University.}

When we have data $x^{(i)} \in \mathbb R^d$
that comes from a mixture of several Gaussians,
the EM algorithm can be applied to fit a mixture model. In this setting, we
usually imagine problems where we have sufficient data to be able to discern
the multiple-Gaussian structure in the data. For instance, this would be the
case if our training set size $n$ was significantly larger than the dimension $d$
of the data.

Now, consider a setting in which $d \gg n$. In such a problem, it might be
difficult to model the data even with a single Gaussian, much less a mixture
of Gaussian. Specifically, since the $n$ data points span only a low-dimensional
subspace of $\mathbb R^d$, if we model the data as Gaussian, and estimate the mean
and covariance using the usual maximum likelihood estimators,
\begin{align*}
    \mu &= \frac 1 n \sum^n_{i=1} x^{(i)}\\
    \Sigma &= \frac 1 n \sum^n_{i=1} (x^{(i)} - \mu)(x^{(i)} - \mu)^\top,
\end{align*}
we would find that the matrix $\Sigma$ is singular. This means that $\Sigma^{-1}$ does not
exist, and $1/|\Sigma|^{1/2} = 1/0$. But both of these terms are needed in computing
the usual density of a multivariate Gaussian distribution. Another way of
stating this difficulty is that maximum likelihood estimates of the parameters
result in a Gaussian that places all of its probability in the affine space
spanned by the data,\footnote{This is the set of points $x$ satisfying $x = \sum^n_{i=1} \alpha_i x^{(i)}$, for some $\alpha_i$'s so that $\sum^n_{i=1} \alpha_i = 1$.} % TYPO: \alpha_1 should be \alpha_i in summation.
and this corresponds to a singular covariance matrix.

More generally, unless $n$ exceeds $d$ by some reasonable amount, the maximum likelihood estimates of the mean and covariance may be quite poor.
Nonetheless, we would still like to be able to fit a reasonable Gaussian model
to the data, and perhaps capture some interesting covariance structure in
the data. How can we do this?

In the next section, we begin by reviewing two possible restrictions on
$\Sigma$ that allow us to fit $\Sigma$ with small amounts of data but neither will give
a satisfactory solution to our problem. We next discuss some properties of
Gaussians that will be needed later; specifically, how to find marginal and
conditonal distributions of Gaussians. Finally, we present the factor analysis
model, and EM for it.

\section{Restrictions of $\Sigma$}
If we do not have sufficient data to fit a full covariance matrix, we may
place some restrictions on the space of matrices $\Sigma$ that we will consider. For
instance, we may choose to fit a covariance matrix $\Sigma$ that is diagonal. In this
setting, the reader may easily verify that the maximum likelihood estimate
of the covariance matrix is given by the diagonal matrix $\Sigma$ satisfying
\[
\Sigma_{jj} =
\frac 1 n \sum^n_{i=1} (x^{(i)}_j - \mu_j)^2.
\]
Thus, $\Sigma_{jj}$ is just the empirical estimate of the variance of the $j$-th coordinate
of the data.

Recall that the contours of a Gaussian density are ellipses. A diagonal
$\Sigma$ corresponds to a Gaussian where the major axes of these ellipses are axis-aligned.

Sometimes, we may place a further restriction on the covariance matrix
that not only must it be diagonal, but its diagonal entries must all be equal.
In this setting, we have $\Sigma = \sigma^2 I$, where $\sigma^2$
is the parameter under our control.
The maximum likelihood estimate of $\sigma^2$ can be found to be:
\[
    \sigma^2 = \frac{1}{nd} \sum^d_{j=1}\sum^n_{i=1}(x^{(i)}_j - \mu_j)^2.
\]
This model corresponds to using Gaussians whose densities have contours
that are circles (in $2$ dimensions; or spheres/hyperspheres in higher dimensions).

If we are fitting a full, unconstrained, covariance matrix $\Sigma$ to data, it is
necessary that $n \ge d + 1$ in order for the maximum likelihood estimate of $\Sigma$
not to be singular. Under either of the two restrictions above, we may obtain
non-singular $\Sigma$ when $n \ge 2$.

However, restricting $\Sigma$ to be diagonal also means modeling the different
coordinates $x_i, x_j$ of the data as being uncorrelated and independent. Often,
it would be nice to be able to capture some interesting correlation structure
in the data. If we were to use either of the restrictions on $\Sigma$ described above,
we would therefore fail to do so. In this set of notes, we will describe the
factor analysis model, which uses more parameters than the diagonal $\Sigma$ and
captures some correlations in the data, but also without having to fit a full
covariance matrix.

\section{Marginals and conditionals of Gaussians}
Before describing factor analysis, we digress to talk about how to find conditional and marginal distributions of random variables with a joint multivariate Gaussian distribution.

Suppose we have a vector-valued random variable
\[
x =
\begin{bmatrix}
    x_1\\
    x_2
\end{bmatrix},
\]
where $x_1 \in \mathbb R^r$, $x_2 \in \mathbb R^s$, and $x \in \mathbb R^{r+s}$.
Suppose $x \sim \mathcal N(\mu , \Sigma)$, where
\[
    \mu = \begin{bmatrix}
        \mu_1\\
        \mu_2        
    \end{bmatrix}
    ,
    \quad
    \Sigma = \begin{bmatrix}
        \Sigma_{11} & \Sigma_{12}\\
        \Sigma_{21} & \Sigma_{22}        
    \end{bmatrix}.
\]
Here, $\mu_1 \in \mathbb R^r$, $\mu_2 \in \mathbb R^s$, $\Sigma_{11} \in \mathbb R^{r \times r}$,
$\Sigma_{12} \in \mathbb R^{r \times s}$, and so on. Note that since
covariance matrices are symmetric, $\Sigma_{12} = \Sigma^\top_{21}$.

Under our assumptions, $x_1$ and $x_2$ are jointly multivariate Gaussian.
What is the marginal distribution of $x_1$? It is not hard to see that $\mathbb E[x_1] = \mu_1$,
and that $\operatorname{Cov}(x_1) = \mathbb E[(x_1 - \mu_1)(x_1 - \mu_1)] = \Sigma_{11}$. To see that the latter is
true, note that by definition of the joint covariance of $x_1$ and $x_2$, we have
that:
\begin{align*}
    \operatorname{Cov}(x) &= \Sigma\\
        &= \begin{bmatrix}
            \Sigma_{11} & \Sigma_{12}\\
            \Sigma_{21} & \Sigma_{22}
        \end{bmatrix}\\
        &= \mathbb E[(x - \mu )(x - \mu)^\top]\\
        &= \mathbb E \begin{bmatrix} 
            \binom{x_1 - \mu_1}{x_2 - \mu_2} & \binom{x_1 - \mu_1}{x_2 - \mu_2}^\top
        \end{bmatrix}\\
    &= \mathbb E
    \begin{bmatrix}
        (x_1 - \mu_1)(x_1 - \mu_1)^\top & (x_1 - \mu_1)(x_2 - \mu_2)^\top\\
        (x_2 - \mu_2)(x_1 - \mu_1)^\top & (x_2 - \mu_2)(x_2 - \mu_2)^\top
    \end{bmatrix}.
\end{align*}
Matching the upper-left subblocks in the matrices in the second and the last
lines above gives the result.

Since marginal distributions of Gaussians are themselves Gaussian, we
therefore have that the marginal distribution of $x_1$ is given by $x_1 \sim \mathcal N (\mu_1, \Sigma_{11})$.
Also, we can ask, what is the conditional distribution of $x_1$ given $x_2$? By
referring to the definition of the multivariate Gaussian distribution, it can
be shown that $x_1 \mid x_2 \sim \mathcal N (\mu_{1 \mid 2}, \Sigma_{1 \mid 2})$, where:
\begin{align}
    \mu_{1 \mid 2} &= \mu_1 + \Sigma_{12}\Sigma^{-1}_22 (x_2 - \mu_2)\label{eq:conditional_mu}\\
    \Sigma_{1 \mid 2} &= \Sigma_{11} - \Sigma_{12}\Sigma^{-1}_22 \Sigma_{21}\label{eq:conditional_Sigma}
\end{align}

When we work with the factor analysis model in the next section, these
formulas for finding conditional and marginal distributions of Gaussians will
be very useful.

\section{The factor analysis model}
In the factor analysis model, we posit a joint distribution on $(x, z)$ as follows,
where $z \in \mathbb R^k$ is a latent random variable:
\begin{align*}
    z &\sim \mathcal N (0, I)\\
    x \mid z &\sim \mathcal N (\mu + \Lambda z, \Psi)
\end{align*}

Here, the parameters of our model are the vector $\mu \in \mathbb R^d$, the matrix
$\Lambda \in \mathbb R^{d \times k}$, and the diagonal matrix $\Psi \in \mathbb R^{d \times d}$.
The value of $k$ is usually chosen to be smaller than $d$.

Thus, we imagine that each datapoint $x^{(i)}$
is generated by sampling a $k$ dimension multivariate Gaussian $z^{(i)}$. Then, it is mapped to a $d$-dimensional
affine space of $\mathbb R^d$ by computing $\mu + \Lambda z^{(i)}$. Lastly, $x^{(i)}$
is generated by adding covariance $\Psi$ noise to $\mu + \Lambda z^{(i)}$.

Equivalently (convince yourself that this is the case), we can therefore
also define the factor analysis model according to
\begin{align*}
    z &\sim \mathcal N(0, I)\\
    \epsilon &\sim \mathcal N (0, \Psi)\\
    x &= \mu + \Lambda z + \epsilon
\end{align*}
where $\epsilon$ and $z$ are independent.

Let's work out exactly what distribution our model defines. Our random
variables $z$ and $x$ have a joint Gaussian distribution
\[
\begin{bmatrix}
    z\\
    x
\end{bmatrix} \sim \mathcal N (\mu_{zx}, \Sigma).
\]
We will now find $\mu_{zx}$ and $\Sigma$.

We know that $\mathbb E[z] = 0$, from the fact that $z \sim \mathcal N (0, I)$. Also, we have
that:
\begin{align*}
    \mathbb E[x] &= \mathbb E[\mu + \Lambda z + \epsilon]
        &= \mu + \Lambda \mathbb E[z] + \mathbb E[\epsilon]
        &= \mu
\end{align*}
Putting these together, we obtain
\[
\mu_{zx} =
\begin{bmatrix}
    \vec{0}\\
    \mu    
\end{bmatrix}
\]
Next, to find $\Sigma$, we need to calculate:
\begin{align*} % DIFF: align.
    \Sigma_{zz} &= \mathbb E[(z - \mathbb E[z])(z - \mathbb E[z])^\top] \tag{the upper-left block of $\Sigma$}\\
    \Sigma_{zx} &= \mathbb E[(z - \mathbb E[z])(x - \mathbb E[x])^\top] \tag{upper-right block}\\
    \Sigma_{xx} &= \mathbb E[(x - \mathbb E[x])(x - \mathbb E[x])^\top] \tag{lower-right block}
\end{align*}

Now, since $z \sim \mathcal N (0, I)$, we easily find that $\Sigma_{zz} = \operatorname{Cov}(z) = I$. Also,
\begin{align*}
    \mathbb E[(z - \mathbb E[z])(x - \mathbb E[x])^\top] &= \mathbb E[z(\mu + \Lambda z + \epsilon - \mu)^\top]\\
        &= \mathbb E[zz^\top]\Lambda^\top + \mathbb E[z\epsilon^\top]\\
        &= \Lambda^\top
\end{align*}
In the last step, we used the fact that $\mathbb E[zz^\top] = \operatorname{Cov}(z)$ (since $z$ has zero
mean), and $\mathbb E[z\epsilon^\top] = \mathbb E[z]\mathbb E[\epsilon^\top] = 0$ (since $z$ and $\epsilon$ are independent, and
hence the expectation of their product is the product of their expectations).
Similarly, we can find $\Sigma_{xx}$ as follows:
\begin{align*}
    \mathbb E[(x - \mathbb E[x])(x - \mathbb E[x])^\top] &= \mathbb E[(\mu + \Lambda z + \epsilon - \mu )(\mu + \Lambda z + \epsilon - \mu)^\top]\\
        &= \mathbb E[\Lambda zz^\top\Lambda^\top + \epsilon z^\top\Lambda^\top + \Lambda z\epsilon^\top + \epsilon \epsilon^\top]\\
        &= \Lambda \mathbb E[zz^\top]\Lambda^\top + \mathbb E[\epsilon \epsilon^\top]\\
        &= \Lambda \Lambda^\top + \Psi
\end{align*}
Putting everything together, we therefore have that
\begin{equation}\label{eq:joint_gaussian}
    \begin{bmatrix}
        z\\
        x
    \end{bmatrix} \sim \mathcal N\left(\begin{bmatrix}
        \vec{0}\\
        \mu
    \end{bmatrix}, \begin{bmatrix}
        I & \Lambda^\top\\
        \Lambda & \Lambda\Lambda^\top + \Psi
    \end{bmatrix} \right).
\end{equation}

Hence, we also see that the marginal distribution of $x$ is given by $x \sim 
N (\mu ,\Lambda \Lambda^\top + \Psi)$. Thus, given a training set ${x^{(i)}
; i = 1, \ldots, n}$, we can write
down the log likelihood of the parameters:
\begin{equation}
    \ell(\mu ,\Lambda , \Psi) = \log \prod^n_{i=1} \frac{1}{(2\pi)^{d/2}|\Lambda \Lambda^\top + \Psi|^{1/2}}\exp\left(-\frac{1}{2} (x^{(i)} - \mu)^\top (\Lambda \Lambda^\top + \Psi)^{-1} (x^{(i)} - \mu)\right)
\end{equation}
To perform maximum likelihood estimation, we would like to maximize this
quantity with respect to the parameters. But maximizing this formula explicitly is hard (try it yourself), and we are aware of no algorithm that does
so in closed-form. So, we will instead use to the EM algorithm. In the next
section, we derive EM for factor analysis.


\section{EM for factor analysis}
The derivation for the E-step is easy. We need to compute $Q_i(z^{(i)}) = p(z^{(i)} \mid x^{(i)}; \mu ,\Lambda , \Psi)$.
By substituting the distribution given in \cref{eq:joint_gaussian}
into the formulas \ref{eq:conditional_mu}-\ref{eq:conditional_Sigma} used for finding the conditional distribution of a
Gaussian, we find that $z^{(i)} \mid x^{(i)}; \mu ,\Lambda , \Psi \sim \mathcal N (\mu_{z^{(i)} \mid x^{(i)}}, \Sigma_{z^{(i)}\mid x^{(i)}})$, where
\begin{align*}
    \mu_{z^{(i)} \mid x^{(i)}} &= \Lambda^\top (\Lambda \Lambda^\top + \Psi)^{-1} (x^{(i)} - \mu)\\
    \Sigma_{z^{(i)} \mid x^{(i)}} &= I - \Lambda^\top (\Lambda \Lambda^\top + \Psi)^{-1}\Lambda
\end{align*}
So, using these definitions for $\mu_{z^{(i)} \mid x^{(i)}}$ and $\Sigma_{z^{(i)} \mid x^{(i)}}$, we have:
\[
    Q_i(z^{(i)}) = \frac{1}{(2\pi)^{k/2}|\Sigma_{z^{(i)} \mid x^{(i)}}|^{1/2}}\exp\left(-\frac{1}{2} (z^{(i)} - \mu_{z^{(i)} \mid x^{(i)}})^\top\Sigma^{-1}_{z^{(i)} \mid x^{(i)}} (z^{(i)} - \mu_{z^{(i)} \mid x^{(i)}})\right)
\]

Let's now work out the M-step. Here, we need to maximize
\begin{equation}\label{eq:m_step_fa_max}
    \sum^n_{i=1}\int_{z^{(i)}} Q_i(z^{(i)})\log \frac{p(x^{(i)}, z^{(i)}; \mu ,\Lambda , \Psi)}{Q_i(z^{(i)})}dz^{(i)}
\end{equation}
with respect to the parameters $\mu ,\Lambda , \Psi$. We will work out only the optimization with respect to $\Lambda$, and leave the derivations of the updates for $\mu$ and $\Psi$
as an exercise to the reader.

We can simplify \cref{eq:m_step_fa_max} as follows:
\begin{align}
    \sum_{i=1}^n &\int_{z^{(i)}} Q_i(z^{(i)}) [\log p(x^{(i)} \mid z^{(i)}; \mu ,\Lambda , \Psi) + \log p(z^{(i)}) - \log Q_i(z^{(i)})] dz^{(i)}\\
    &= \sum_{i=1}^n \mathbb E_{z^{(i)} \sim Q_i} [\log p(x^{(i)} \mid z^{(i)}; \mu ,\Lambda , \Psi) + \log p(z^{(i)}) - \log Q_i(z^{(i)})]
\end{align}
Here, the ``$z^{(i)} \sim Q_i$'' subscript indicates that the expectation is with respect
to $z^{(i)}$ drawn from $Q_i$. In the subsequent development, we will omit this
subscript when there is no risk of ambiguity. Dropping terms that do not
depend on the parameters, we find that we need to maximize:
\begin{align*}
    \sum_{i=1}^n &\mathbb E[\log p(x^{(i)} \mid z^{(i)}; \mu ,\Lambda , \Psi)]\\
        &= \sum^n_{i=1} \mathbb E\left[\log \frac{1}{(2\pi)^{d/2}|\Psi|^{1/2}}\exp\left(-\frac{1}{2} (x^{(i)} - \mu - \Lambda z^{(i)})^\top\Psi^{-1}(x^{(i)} - \mu - \Lambda z^{(i)})\right)\right]\\
        &= \sum^n_{i=1} \mathbb E\left[-\frac{1}{2}\log |\Psi| - \frac{n}{2}\log(2\pi) -\frac{1}{2} (x^{(i)} - \mu - \Lambda z^{(i)})^\top\Psi^{-1}(x^{(i)} - \mu - \Lambda z^{(i)})\right]
\end{align*}
Let's maximize this with respect to $\Lambda$. Only the last term above depends
on $\Lambda$. Taking derivatives, and using the facts that $\operatorname{tr}(a)=a$ (for $a \in \mathbb R$),
$\operatorname{tr}(AB) = \operatorname{tr}(BA)$, and $\nabla_A\operatorname{tr}(ABA^\top C) = CAB + C^\top AB^\top$, we get:
\begin{align*}
    \nabla_\Lambda &\sum^n_{i=1} -\mathbb E\left[\frac{1}{2}(x^{(i)} - \mu - \Lambda z^{(i)})^\top\Psi^{-1}(x^{(i)} - \mu - \Lambda z^{(i)})\right]\\
        &= \sum^n_{i=1} \nabla_\Lambda \mathbb E\left[-\operatorname{tr}(\frac{1}{2} z^{(i)^\top}\Lambda^\top\Psi^{-1}\Lambda z^{(i)}) + \operatorname{tr}(z^{(i)^\top}\Lambda^\top\Psi^{-1}(x^{(i)} - \mu))\right]\\
        &= \sum^n_{i=1} \nabla_\Lambda \mathbb E\left[-\operatorname{tr}(\frac{1}{2} \Lambda^\top \Psi^{-1} \Lambda z^{(i)} z^{(i)^\top}) + \operatorname{tr}(\Lambda^\top\Psi^{-1}(x^{(i)} - \mu)z^{(i)^\top})\right]\\
        &= \sum^n_{i=1} \mathbb E\left[-\Psi^{-1} \Lambda z^{(i)} z^{(i)^\top} + \Psi^{-1}(x^{(i)} - \mu)z^{(i)^\top}\right]
\end{align*}
Setting this to zero and simplifying, we get:
\[
    \sum^n_{i=1} \Lambda \mathbb E_{z^{(i)} \sim Q_i} \left[z^{(i)}z^{(i)^\top}\right] = \sum^n_{i=1} (x^{(i)} - \mu) \mathbb E_{z^{(i)}\sim Q_i}\left[z^{(i)^\top}\right]
\]
Hence, solving for $\Lambda$, we obtain
\begin{equation}\label{eq:Lambda}
    \Lambda = \left(\sum^n_{i=1} (x^{(i)} - \mu)\mathbb E_{z^{(i)}\sim Q_i}\left[z^{(i)^\top}\right]\right) \left(\sum^n_{i=1} \mathbb E_{z^{(i)}\sim Q_i}\left[z^{(i)}z^{(i)^\top}\right]\right)^{-1}
\end{equation}
It is interesting to note the close relationship between this equation and the
normal equation that we'd derived for least squares regression,
\[
\text{``}\theta^\top = (y^\top X)(X^\top X)^{-1}\text{.''}
\]
The analogy is that here, the $x$'s are a linear function of the $z$'s (plus noise).
Given the ``guesses'' for $z$ that the E-step has found, we will now try to
estimate the unknown linearity $\Lambda$ relating the $x$'s and $z$'s. It is therefore
no surprise that we obtain something similar to the normal equation. There
is, however, one important difference between this and an algorithm that
performs least squares using just the ``best guesses'' of the $z$'s; we will see
this difference shortly.

To complete our M-step update, let's work out the values of the expectations in \cref{eq:Lambda}. From our definition of $Q_i$ being Gaussian with mean
$\mu_{z^{(i)} \mid x^{(i)}}$ and covariance $\Sigma_{z^{(i)} \mid x^{(i)}}$, we easily find
\begin{align*}
    \mathbb E_{z^{(i)}\sim Q_i}\left[z^{(i)^\top}\right] &= \mu^\top_{z^{(i)} \mid x^{(i)}}\\
    \mathbb E_{z^{(i)}\sim Q_i}\left[z^{(i)}z^{(i)^\top}\right] &= \mu_{z^{(i)} \mid x^{(i)}}\mu^\top_{z^{(i)} \mid x^{(i)}} + \Sigma_{z^{(i)} \mid x^{(i)}}
\end{align*}
The latter comes from the fact that, for a random variable $Y$, $\operatorname{Cov}(Y) = \mathbb E[Y Y^\top] - \mathbb E[Y]\mathbb E[Y]^\top$, and hence
$\mathbb E[Y Y^\top] = \mathbb E[Y]\mathbb E[Y]^\top + \operatorname{Cov}(Y)$. Substituting this back into \cref{eq:Lambda}, we get the M-step update for $\Lambda$:
\begin{equation}\label{eq:Lambda_m_step}
    \Lambda = \left(\sum^n_{i=1} (x^{(i)} - \mu )\mu^\top_{z^{(i)} \mid x^{(i)}}\right) \left(\sum^n_{i=1} \mu_{z^{(i)} \mid x^{(i)}}\mu^\top_{z^{(i)} \mid x^{(i)}} + \Sigma_{z^{(i)} \mid x^{(i)}}\right)^{-1}
\end{equation}
It is important to note the presence of the $\Sigma_{z^{(i)} \mid x^{(i)}}$ on the right hand side of
this equation. This is the covariance in the posterior distribution $p(z^{(i)} \mid x^{(i)})$
of $z^{(i)}$ given $x^{(i)}$, % TYPO: "give" -> "given"
and the M-step must take into account this uncertainty
about $z^{(i)}$ in the posterior. A common mistake in deriving EM is to assume
that in the E-step, we need to calculate only expectation $\mathbb E[z]$ of the latent
random variable $z$, and then plug that into the optimization in the M-step
everywhere $z$ occurs. While this worked for simple problems such as the
mixture of Gaussians, in our derivation for factor analysis, we needed $\mathbb E[zz^\top
]$ as well as $\mathbb E[z]$; and as we saw, % TYPO: missing "as"
$\mathbb E[zz^\top]$ and $\mathbb E[z]\mathbb E[z]^\top$ differ by the quantity $\Sigma_{z \mid x}$.
Thus, the M-step update must take into account the covariance of $z$ in the
posterior distribution $p(z^{(i)} \mid x^{(i)})$.

Lastly, we can also find the M-step optimizations for the parameters $\mu$ 
and $\Psi$. It is not hard to show that the first is given by
\[
    \mu = \frac{1}{n} \sum_{i=1}^n x^{(i)}.
\]
Since this doesn't change as the parameters are varied (i.e., unlike the update
for $\Lambda$, the right hand side does not depend on $Q_i(z^{(i)}) = p(z^{(i)} \mid x^{(i)}; \mu ,\Lambda , \Psi)$,
which in turn depends on the parameters), this can be calculated just once
and needs not be further updated as the algorithm is run. Similarly, the
diagonal $\Psi$ can be found by calculating
\[
\Phi = \frac{1}{n}
\sum^n_{i=1} x^{(i)}x^{(i)^\top} - x^{(i)}\mu^\top_{z^{(i)} \mid x^{(i)}}\Lambda^\top - \Lambda \mu_{z^{(i)} \mid x^{(i)}} x^{(i)^\top} + \Lambda \left(\mu_{z^{(i)} \mid x^{(i)}}\mu^\top_{z^{(i)} \mid x^{(i)}} + \Sigma_{z^{(i)} \mid x^{(i)}}\right)\Lambda^\top,
\]    
and setting $\Psi_{ii} = \Phi_{ii}$ (i.e., letting $\Psi$ be the diagonal matrix containing only
the diagonal entries of $\Phi$).
