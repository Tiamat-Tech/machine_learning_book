\titlespacing*{\part}{0pt}{-20pt}{30pt} % to fix table and plot together on the first page (SEE RESTORE AT BOTTOM)
\titlespacing*{\chapter}{0pt}{-10pt}{30pt}

\part{Principal Components Analysis}
\label{part:pca}

\marginnote{From CS229 Spring 2021, Andrew Ng, Moses Charikar \& Christopher R\'e, Stanford University.}

In our discussion of factor analysis, we gave a way to model data $x \in \mathbb R^d$ as
``approximately'' lying in some $k$-dimension subspace, where $k \ll d$. Specifically, we imagined that each point $x^{(i)}$ was created by first generating some $z^{(i)}$
lying in the $k$-dimension affine space ${\Lambda z + \mu; z \in \mathbb R^k}$, and then adding
$\Psi$-covariance noise. Factor analysis is based on a probabilistic model, and
parameter estimation used the iterative EM algorithm.

In this set of notes, we will develop a method, \textbf{Principal Components
Analysis} (PCA), that also tries to identify the subspace in which the data
approximately lies. However, PCA will do so more directly, and will require
only an eigenvector calculation (easily done with the \texttt{eig} function in Matlab),
and does not need to resort to EM.

Suppose we are given a dataset ${x^{(i)}; i = 1, \ldots, n}$ of attributes of $n$ different types of automobiles, such as their maximum speed, turn radius, and
so on. Let $x^{(i)} \in \mathbb R^d$ for each $i$ ($d \ll n$). But unknown to us, two different
attributes---some $x_i$ and $x_j$---respectively give a car's maximum speed measured in miles per hour, and the maximum speed measured in kilometers per
hour. These two attributes are therefore almost linearly dependent, up to
only small differences introduced by rounding off to the nearest mph or kph.
Thus, the data really lies approximately on an $n - 1$ dimensional subspace.
How can we automatically detect, and perhaps remove, this redundancy?

For a less contrived example, consider a dataset resulting from a survey of
pilots for radio-controlled helicopters, where $x^{(i)}_1$
is a measure of the piloting
skill of pilot $i$, and $x^{(i)}_2$
captures how much he/she enjoys flying. Because
RC helicopters are very difficult to fly, only the most committed students,
ones that truly enjoy flying, become good pilots. So, the two attributes
$x_1$ and $x_2$ are strongly correlated. Indeed, we might posit that that the
data actually lies along some diagonal axis (the $u_1$ direction) capturing the % TYPO: "likes"
intrinsic piloting ``karma'' of a person, with only a small amount of noise
lying off this axis. (See figure.) How can we automatically compute this $u_1$
direction?
% TODO: Figure.

We will shortly develop the PCA algorithm. But prior to running PCA
per se, typically we first preprocess the data by normalizing each feature
to have mean $0$ and variance $1$. We do this by subtracting the mean and
dividing by the empirical standard deviation:
\[
    x^{(i)}_j \leftarrow \frac{x^{(i)}_j - \mu_j}{\sigma_j}
\]
where $\mu_j = \frac{1}{n} \sum^n_{i=1} x^{(i)}_j$
and $\sigma^2_j = \frac{1}{n} \sum_{i=1}^n (x^{(i)}_j - \mu_j)^2$ are the mean variance of
feature $j$, respectively.

Subtracting $\mu_j$ zeros out the mean and may be omitted for data known
to have zero mean (for instance, time series corresponding to speech or other
acoustic signals). Dividing by the standard deviation $\sigma_j$ rescales each coordinate to have unit variance, which ensures that different attributes are all
treated on the same ``scale.'' For instance, if $x_1$ was cars' maximum speed in
mph (taking values in the high tens or low hundreds) and $x_2$ were the number of seats (taking values around 2-4), then this renormalization rescales
the different attributes to make them more comparable. This rescaling may
be omitted if we had a priori knowledge that the different attributes are all
on the same scale. One example of this is if each data point represented a
grayscale image, and each $x^{(i)}_j$
took a value in $\{0, 1, \ldots, 255\}$ corresponding
to the intensity value of pixel $j$ in image $i$.

Now, having normalized our data, how do we compute the ``major axis
of variation'' $u$---that is, the direction on which the data approximately lies?
One way is to pose this problem as finding the unit vector $u$ so that when
the data is projected onto the direction corresponding to $u$, the variance of
the projected data is maximized. Intuitively, the data starts off with some
amount of variance/information in it. We would like to choose a direction $u$
so that if we were to approximate the data as lying in the direction/subspace
corresponding to $u$, as much as possible of this variance is still retained.
Consider the following dataset, on which we have already carried out the
normalization steps:

% TODO: Figure.

Now, suppose we pick $u$ to correspond the the direction shown in the
figure below. The circles denote the projections of the original data onto this
line.

% TODO: Figure, projection.

We see that the projected data still has a fairly large variance, and the
points tend to be far from zero. In contrast, suppose had instead picked the
following direction:

% TODO: Figure.

Here, the projections have a significantly smaller variance, and are much
closer to the origin.

We would like to automatically select the direction u corresponding to
the first of the two figures shown above. To formalize this, note that given a
unit vector $u$ and a point $x$, the length of the projection of $x$ onto $u$ is given
by $x^\top u$. I.e., if $x^{(i)}$
is a point in our dataset (one of the crosses in the plot),
then its projection onto $u$ (the corresponding circle in the figure) is distance
$x^\top u$ from the origin. Hence, to maximize the variance of the projections, we
would like to choose a unit-length $u$ so as to maximize:
\begin{align*}
    \frac{1}{n} \sum_{i=1}^n (x^{(i)^\top} u)^2 &= \frac{1}{n} \sum_{i=1}^n u^\top x^{(i)} x^{(i)^\top} u\\
        &= u^\top \left( \frac{1}{n} \sum_{i=1}^n x^{(i)} x^{(i)^\top} \right) u
\end{align*}
We easily recognize that the maximizing this subject to $\lVert u \rVert^2 = 1$ gives the
principal eigenvector of $\Sigma = \frac{1}{n}\sum^n_{i=1} x^{(i)}x^{(i)^\top}$, which is just the empirical
covariance matrix of the data (assuming it has zero mean).\footnote{
If you haven't seen this before, try using the method of Lagrange multipliers to maximize $u^\top \Sigma u$ subject to that $u^\top u = 1$.
You should be able to show that $\Sigma u = \lambda u$, for some
$\lambda$, which implies $u$ is an eigenvector of $\Sigma$, with eigenvalue $\lambda$.}

To summarize, we have found that if we wish to find a 1-dimensional
subspace with with to approximate the data, we should choose $u$ to be the
principal eigenvector of $\Sigma$. More generally, if we wish to project our data
into a $k$-dimensional subspace ($k < d$), we should choose $u_1, \ldots, u_k$ to be the
top $k$ eigenvectors of $\Sigma$. The $u_i$'s now form a new, orthogonal basis for the
data.\footnote{Because $\Sigma$ is symmetric, the $u_i$'s will (or always can be chosen to be) orthogonal toeach other.}

Then, to represent $x^{(i)}$ in this basis, we need only compute the corresponding vector
\[
y^{(i)} = \begin{bmatrix}
    u_1^\top x^{(i)}\\
    u_2^\top x^{(i)}\\
    \vdots\\
    u_k^\top x^{(i)}\\
\end{bmatrix} \in \mathbb R^k.
\]
Thus, whereas $x^{(i)} \in \mathbb R^d$, the vector $y^{(i)}$ now gives a lower, $k$-dimensional,
approximation/representation for $x^{(i)}$. PCA is therefore also referred to as
a \textbf{dimensionality reduction} algorithm. The vectors $u_1, \ldots, u_k$ are called
the first $k$ \textbf{principal components} of the data.

\paragraph{Remark} Although we have shown it formally only for the case of $k = 1$,
using well-known properties of eigenvectors it is straightforward to show that
of all possible orthogonal bases $u_1, \ldots, u_k$, the one that we have chosen maximizes $\sum_i \lVert y^{(i)} \rVert_2^2$. Thus, our choice of a basis preserves as much variability
as possible in the original data.

In problem set 4, you will see that PCA can also be derived by picking
the basis that minimizes the approximation error arising from projecting the
data onto the $k$-dimensional subspace spanned by them.

PCA has many applications; we will close our discussion with a few examples. First, compression---representing $x^{(i)}$'s with lower dimension $y^{(i)}$'s---is an
obvious application. If we reduce high dimensional data to $k = 2$ or $3$ dimensions, then we can also plot the $y^{(i)}$'s to visualize the data. For instance,
if we were to reduce our automobiles data to $2$ dimensions, then we can plot
it (one point in our plot would correspond to one car type, say) to see what
cars are similar to each other and what groups of cars may cluster together.

Another standard application is to preprocess a dataset to reduce its
dimension before running a supervised learning learning algorithm with the
$x^{(i)}$'s as inputs. Apart from computational benefits, reducing the data's
dimension can also reduce the complexity of the hypothesis class considered
and help avoid overfitting (e.g., linear classifiers over lower dimensional input
spaces will have smaller VC dimension).

Lastly, as in our RC pilot example, we can also view PCA as a noise
reduction algorithm. In our example it, estimates the intrinsic ``piloting
karma'' from the noisy measures of piloting skill and enjoyment. In class, we
also saw the application of this idea to face images, resulting in \textbf{eigenfaces}
method. Here, each point $x^{(i)} \in \mathbb R^{100 \times 100}$ was a $10000$ dimensional vector,
with each coordinate corresponding to a pixel intensity value in a $100\times100$
image of a face. Using PCA, we represent each image $x^{(i)}$ with a much lowerdimensional $y^{(i)}$.
In doing so, we hope that the principal components we
found retain the interesting, systematic variations between faces that capture
what a person really looks like, but not the ``noise'' in the images introduced
by minor lighting variations, slightly different imaging conditions, and so on.
We then measure distances between faces $i$ and $j$ by working in the reduced
dimension, and computing $\lVert y^{(i)} - y^{(j)} \rVert_2$. This resulted in a surprisingly good
face-matching and retrieval algorithm.
